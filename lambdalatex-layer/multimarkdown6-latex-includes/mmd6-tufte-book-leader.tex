%
% 	tufte-latex book for MultiMarkdown
%		http://code.google.com/p/tufte-latex/
%
%	Creates a basic handout emulating part of Edward Tufte's style
%	from some of his books
%
%	* Only h1 and h2 are valid
%	* \citep may be better than \cite
%	* \autoref doesn't work properly, may get better results with \ref
%	* footnotes don't work inside of tables
%


\documentclass{tufte-book}
%\documentclass[justified]{tufte-handout}


% Use default packages for memoir setup

\usepackage{fancyvrb}			% Allow \verbatim et al. in footnotes
\usepackage{graphicx}			% To enable including graphics in pdf's
\usepackage{booktabs}			% Better tables
\usepackage{tabulary}			% Support longer table cells
\usepackage[utf8]{inputenc}		% For UTF-8 support
\usepackage[T1]{fontenc}		% Use T1 font encoding for accented characters
\usepackage{xcolor}				% Allow for color (annotations)
\usepackage{listings}			% Allow for source code highlighting
\usepackage{subscript}
\usepackage[normalem]{ulem}		% Support strikethrough

% CriticMarkup Support
\usepackage{soul}
\usepackage{xargs}
\usepackage{todonotes}
\newcommandx{\cmnote}[2][1=]{\linespread{1.0}\todo[linecolor=red,backgroundcolor=red!25,bordercolor=red,#1]{#2}}

% Use \ul instead of \underline since we are using soul
\let\underline\ul


\usepackage[acronym]{glossaries}
\glstoctrue
\makeglossaries
\makeindex


% Configure default metadata to avoid errors
%
%	Configure default metadata in case it's missing to avoid errors
%

\def\myauthor{Author}
\def\defaultemail{}
\def\defaultposition{}
\def\defaultdepartment{}
\def\defaultaddress{}
\def\defaultphone{}
\def\defaultfax{}
\def\defaultweb{}
\def\defaultaffiliation{}

\def\mytitle{Title}
\def\subtitle{}
\def\mykeywords{}


\def\bibliostyle{plain}
% \def\bibliocommand{}

\def\myrecipient{}

% Overwrite with your own if desired
%\input{ftp-metadata}



